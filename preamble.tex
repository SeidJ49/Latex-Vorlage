\documentclass[
BCOR=5mm,           % Binderkorrektur von 5mm vorsehen
fontsize=11pt,      % Schriftgr��e 11 Punkte
oneside,            % Einseitig
parskip,            % Paragraphen nicht einrücken
headsepline,        % Kopfzeile nach unten durch Linie abgrenzen (scrheadings)
%footbotline,       % Fu�zeile nach unten durch Linie abgrenzen (scrheadings)
plainheadsepline,   % Kopfzeile nach unten durch Linie abgrenzen (scrplain)
plainfootbotline,   % Fu�zeile nach unten durch Linie abgrenzen (scrplain)
%headtopline,       % Kopfzeile nach oben durch Linie abgrenzen (scrheadings)
footsepline,        % Fu�zeile nach oben durch Linie abgrenzen (scrheadings)
plainheadtopline,   % Kopfzeile nach oben durch Linie abgrenzen (scrplain)
plainfootsepline,   % Fu�zeile nach oben durch Linie abgrenzen (scrplain)
]{scrbook}          % Koma-Script Klasse zum setzen eines Buchs


\usepackage{lmodern}

% Die "Standard-Header" f�r deutsche Dokumente
\usepackage[latin1]{inputenc}    % ISO-8859-1 bzw. Latin1 als Encoding
%\usepackage[T1]{fontenc}         % T1 Schriften verwenden (sieht besser aus)
\usepackage[ngerman]{babel}      % Neue deutsche Rechtschreibung 

% "Sch�nere" Schriften einbinden
%\usepackage{mathpazo}            % Serifen-Font mit passendem Math-Font
%\usepackage[scaled=.95]{helvet}  % Serifenloser Font passend zu mathpazo
%\usepackage{courier}             % "Sch�nerer" Festbreiten-Font

% Koma-Script Paket zum setzen vom Kopf- und Fu�zeilen einbinden
%\usepackage{scrpage2}
\usepackage{scrlayer-scrpage}
\RequirePackage{scrlfile}
\ReplacePackage{scrpage2}{scrlayer-scrpage}
\usepackage{substr}

%\pagestyle{scrheadings}
\clearpairofpagestyles
\automark[section]{chapter}

\ohead[\sffamily\scshape\bfseries\large\headmark]
{\sffamily\scshape\bfseries\large\headmark}

\ofoot[\sffamily\thepage]{\sffamily\thepage}

\usepackage{listings}
\lstset{
captionpos=b,                % Beschriftung unterhalb (bottom)
numbers=left,                % Zeilennummern links
frame=trbl,                  % Rahmen zeichnen (top, right, bottom, left)
basicstyle=\small\ttfamily,  % Festbreitenschrift verwenden (small)
language=Java                % Sprache auf Java einstellen
}

\usepackage[ngerman]{translator}

\usepackage[
nonumberlist, % Keine Seitenzahlen anzeigen
acronym,      % Abkürzungsverzeichnis erstellen
toc,          % In Inhaltsverzeichnis aufnehmen
%section       % Verzeichniseintrag als Section
]{glossaries}

% Ein eigenes Verzeichnis definieren (Smbolverzeichnis)
% Das Stichwort- und Abkürzungsverzeichnis wird analog vordefiniert
% Siehe makeindex Aufrufe - Hier werden die Dateiendungen festgelegt
\newglossary[slg]{symbolslist}{syi}{syg}{Symbolverzeichnis}

\makeglossaries


% Paket zum generieren von Blindtext
\usepackage{blindtext}

% Paket zum Einbinden von Bildern
\usepackage{graphicx}

%              
% WORKAROUND, damit lstlistoflistings funktioniert:
% Quelle: http://www.komascript.de/node/477
%
\makeatletter
\@ifundefined{float@listhead}{}{%
    \renewcommand*{\lstlistoflistings}{%
        \begingroup
    	    \if@twocolumn
                \@restonecoltrue\onecolumn
            \else
                \@restonecolfalse
            \fi
            \float@listhead{\lstlistlistingname}%
            \setlength{\parskip}{\z@}%
            \setlength{\parindent}{\z@}%
            \setlength{\parfillskip}{\z@ \@plus 1fil}%
            \@starttoc{lol}%
            \if@restonecol\twocolumn\fi
        \endgroup
    }%
}
\makeatother
