\chapter{Grundlagen}
In diesem Kapitel werden die wichtigen Grundlagen f�r diese
Aufgabenstellung erl�utert.

\section{Die Programmiersprache Java}
Da das Projekt in Java\index{Java} realisiert wird, wird hier ein kurzer
�berblick gegeben. Java kann auch Rechnen, somit ist die L�sung f�r Gleichung
\ref{equ:plus} auf Seite \pageref{equ:plus} nicht weit entfernt \cite{PAPULA}
\cite{MARTIN}.
\begin{equation}
\label{equ:plus}
x = 1 + y\ :\ y = 7
\end{equation}

\subsection{Klassen}
In Listing \ref{lst:javaclass} auf Seite \pageref{lst:javaclass} ist
eine einfache Java Klasse\index{Klasse} zu sehen. In den letzten Jahren
hat sich die Bezeichnung \textbf{\acrshort{acr:POJO}} f�r einfache Klassen
eingeb�rgert. Eine ausf�hrliche Einf�hrung in Java\index{Java} und die
\textbf{Objekt Orientierte Programmierung (OOP)} ist in \emph{Java als
erste Programmiersprache} von Joachim Goll, Cornelia Heinisch und Frank
M�ller enthalten \cite{GOLL}.
\begin{lstlisting}[caption=Eine einfache Java Klasse,label=lst:javaclass]
class Simple
{
        private String text;
        
        public Simple(String text) {
                this.text = text;
        }
        
        public void printText() {
                System.out.println(text);
        }
}
\end{lstlisting}

\subsection{Java Logo}
In Abbildung \ref{fig:javalogo} auf Seite \pageref{fig:javalogo}
ist das Java\index{Java} Logo abgebildet. Die \textbf{\acrshort{acr:JRE}},
sowie der \textbf{\acrshort{acr:JDK}}, verwenden dieses Logo\index{Logo} an
vielen Stellen. Es ist auch sonst in vielen Programmen und
Internet\index{Internet} Seiten zu sehen.

\begin{figure}[hb]
\centering
\includegraphics[scale=0.8]{bilder/javalogo.jpg}
\caption{Das Java Logo}
\label{fig:javalogo}
\end{figure}

Es gibt aber auch andere Logos, die mit Java in Zusammenhang stehen.
Ein Beispiel ist das Java Maskottchen Duke, das in Abbildung \ref{fig:duke}
auf Seite \pageref{fig:duke} gezeigt wird.

\begin{figure}[hb]
\centering
\includegraphics[scale=0.6]{bilder/duke.jpg}
\caption{Duke, das Java Maskottchen}
\label{fig:duke}
\end{figure}

\subsection{Zugriffsschutz}
Um Teile einer Klasse gegen unbefugten Zugriff zu sch�tzen besitzt
Java\index{Java} die in Tabelle \ref{tab:modifier} auf Seite
\pageref{tab:modifier} enthaltenen \textbf{Modifizierer}\index{Modifizierer}.

\begin{table}[hb]
\centering
\begin{tabular}{|c|c|}
\hline
\textbf{Modifier} & \textbf{Wirkung}\\
\hline
\hline
\texttt{private} & Privat\\
\hline
\texttt{public} & �ffentlich\\
\hline
\texttt{protected} & Eingeschr�nkt\\
\hline
\texttt{final} & Nicht �nderbar\\
\hline
... & ...\\
\hline
\end{tabular}
\caption{Modifizierer f�r den Zugriffsschutz}
\label{tab:modifier}
\end{table}

Das Prinzip des \textbf{Information Hiding}\index{Information Hiding} schreibt
vor dass man den Zugriff soweit wie M�glich einschr�nkt.

\subsection{Die Klasse \texttt{Object}}
In Java\index{Java} erben alle Klassen\index{Klasse} automatisch von der Klasse
\texttt{Object}\index{Objekt}. Einige wichtige Methoden\index{Methode} der
Klasse \texttt{Object} sind in Tabelle \ref{tab:object} auf Seite
\pageref{tab:object} aufgelistet.

\begin{table}[hb]
\centering
\begin{tabular}{|c|c|}
\hline
\textbf{Methode} & \textbf{Funktion}\\
\hline
\hline
\texttt{toString()} & Darstellung als \texttt{String}\\
\hline
\texttt{equals()} & Vergleiche mit anderen Objekten\\
\hline
\texttt{hashCode()} & Hashcode erzeugen (Siehe \texttt{Hashtable})\\
\hline
\texttt{clone()} & Exakte Kopie erzeugen\\
\hline
... & ...\\
\hline
\end{tabular}
\caption{Wichtige Methoden der Klasse \texttt{Object}}
\label{tab:object}
\end{table}

Es sollten im Normalfall mindestens die Methoden\index{Methode}
\texttt{toString()} und \texttt{equals()} �berschrieben werden. Das
�berschreiben von \texttt{clone()} kann aber -- je nach Anwendung -- auch Sinn
machen. Beim �berschreiben von Methoden sollte man die
Annotation\index{Annotation} \texttt{@Over- ride} verwenden, da sie einen
davon abh�lt die Methode versehentlich nur zu �berladen.

