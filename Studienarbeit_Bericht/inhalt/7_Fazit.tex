\chapter{Fazit}

Die Implementierung des A*-Algorithmus in C++ erm�glichte ein tieferes Verst�ndnis der Funktionsweise des Algorithmus. Durch die praktische Umsetzung konnten die einzelnen Schritte des Algorithmus detailliert nachvollzogen und seine Funktionsweise verinnerlicht werden. Die Wahl der Heuristik und die effiziente Nutzung der Datenstrukturen erwiesen sich als entscheidend f�r die Leistungsf�higkeit des Algorithmus. Ein sorgf�ltige Auswahl der Heuristik kann die Effizienz des A*-Algorithmus erheblich verbessern und eine schnellere Suche nach dem optimalen Pfad erm�glichen. 

Die Evaluierung der Performance des A*-Algorithmus auf verschiedenen Graphen f�hrte zu interessanten Erkenntnissen. Es zeigte sich, dass die Laufzeit des Algorithmus stark von der Gr��e und Struktur des Graphen abh�ngt. In einfachen Graphen mit wenigen Knoten und Kanten konnte der A*-Algorithmus schnell eine optimale L�sung finden. In komplexeren Graphen mit vielen Knoten und Kanten stieg die Laufzeit jedoch deutlich an, da der Algorithmus eine umfangreichere Suche durchf�hren musste.

Allerdings wurden auch einige Grenzen des A*-Algorithmus aufgezeigt. In bestimmten Situationen, wie z.B. bei sehr komplexen Graphen ohne klar definierte Ziele, verlor der Algorithmus an Effizienz und n�herte sich dem optimalen Pfad nur langsam an. In solchen F�llen k�nnten alternative Routenplanungsalgorithmen, wie der Bellman-Ford-Algorithmus oder die Breitensuche bessere Ergebnisse liefern.

Abschlie�end ist festzuhalten, dass die Implementierung und Evaluierung des Routen\-planungsalgorithmus eine wertvolle Erfahrung war und ein fundiertes Verst�ndnis dieses leistungsf�higen Routen\-planungsalgorithmus erm�glicht hat. Die gewonnenen Erkenntnisse �ber die Leistungsf�higkeit und der Vergleich mit anderen Routenplanungsalgorithmen liefern wertvolle Informationen f�r zuk�nftige Anwendungen und Optimierungen des A*-Algorithmus.
\newpage
