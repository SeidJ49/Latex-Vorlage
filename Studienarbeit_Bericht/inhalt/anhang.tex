\chapter{Anhang}
Im Anhang dieser Studienarbeit finden Sie weitere Informationen und Materialien, die zur Vertiefung des Themas dienen.

\section{Code zum lesen der GraphML-Datei}
Hier finden Sie den vollst�ndigen Quellcode f�r das Einlesen der GraphML-Datei in der Programmiersprache C++. Der Quellcode enth�lt Kommentare und Erkl�rungen, die den Algorithmus und seine Funktionsweise veranschaulichen.

\subsection{car\_readFile.hpp}
\begin{lstlisting}
#ifndef PJ_RA_CAR_READFILE_HPP
#define PJ_RA_CAR_READFILE_HPP

#include <vector>
#include <string>
#include <fstream>
#include <iostream>
#include <sstream>
#include <algorithm>
#include <unordered_map>

using namespace std;

// Default filename for reading graph data (test.graphml).
constexpr char c_filename[] = "test.graphml";

// Data structure representing a node in the graph.
struct NodeData
{
	NodeData() : m_id{0U}, m_x{0.0}, m_y{0.0}
	{
	}
	
	NodeData(uint16_t id, float x, float y, vector<int> connectedNodes)
	{
		this->m_id = id;
		this->m_x = x;
		this->m_y = y;
		this->connectedNodes = connectedNodes;
	}
	
	uint16_t m_id;              // The ID of the node in the graph.
	float m_x;                  // The x-coordinate of the node.
	float m_y;                  // The y-coordinate of the node.
	vector<int> connectedNodes; // A list of IDs of nodes connected to this node.
	
	// Custom comparison operator used for sorting nodes by their ID.
	bool operator <(const NodeData &other) const
	{
		return (this->m_id < other.m_id);
	}
};

// Class for reading data from a graph file.
class ReadFile
{
	public:
	ReadFile();                     // Default constructor with default filename.
	ReadFile(string filename);      // Constructor with a custom filename.
	
	void readFile();                // Read the graph data from the file.
	
	void setFilename(string filename);               // Set a new filename for reading graph data.
	unordered_map<uint16_t, NodeData> getNodes();    // Get the map of nodes read from the file.
	void exportMapToInl();                          // Export the node data to an inline file (not implemented).
	
	protected:
	NodeData m_currentNode;             // Current node being read from the file.
	unordered_map<uint16_t, NodeData> m_nodes;    // Map to store all the nodes read from the file.
	
	fstream graphMlFile;            // Filestream for reading the graph data.
	string m_filename;              // Filename of the graph file being read.
	string m_currentLine;           // Current line being processed while reading the file.
	
	// Read node data from the file and store it in the current node.
	void readNodeDataFromFile();
	
	// Read edge data from the file and update connectedNodes of nodes accordingly.
	void readEdgeDataFromFile();
};

#endif

\end{lstlisting}

\subsection{car\_readFile.cpp}
\begin{lstlisting}
#include "car_readFile.hpp"

// Default constructor sets the filename to the default value (test.graphml).
ReadFile::ReadFile()
{
	this->m_filename = c_filename;
}

// Constructor with a custom filename provided by the user.
ReadFile::ReadFile(string filename)
{
	this->m_filename = filename;
}

// Read the graph data from the file using the specified filename.
void ReadFile::readFile()
{
	// Open the graphML file for reading.
	graphMlFile.open(m_filename);
	
	// Check if the file was successfully opened.
	if (!graphMlFile)
	{
		cerr << "Fehler beim �ffnen von " << m_filename << "!" << endl;
		throw;
	}
	
	// Read the file line by line and process the data.
	while (getline(graphMlFile, m_currentLine))
	{
		// If the line contains "<node", read node data from the file.
		if (m_currentLine.find("<node") != string::npos)
		{
			readNodeDataFromFile();
			m_nodes[m_currentNode.m_id] = m_currentNode;
		}
		// If the line contains "<edge", read edge data from the file.
		else if (m_currentLine.find("<edge", 0) != string::npos)
		{
			readEdgeDataFromFile();
		}
		// Otherwise, continue to the next line.
		else
		{
			continue;
		}
	}
}

// Read node data from the file and store it in the current node.
void ReadFile::readNodeDataFromFile()
{
	// Continue reading until reaching the end of the current node data.
	do
	{
		// If the line contains "<node", extract the node ID.
		if (m_currentLine.find("<node", 0) != string::npos)
		{
			string searchString = "id=\"";
			string endString = "\">";
			
			size_t startPos = m_currentLine.find(searchString) + searchString.length();
			size_t endPos = m_currentLine.find(endString);
			string value = m_currentLine.substr(startPos, endPos - startPos);
			
			m_currentNode.m_id = stoi(value);
		}
		// If the line contains "<data key", extract the x and y values.
		else if (m_currentLine.find("<data key", 0) != string::npos)
		{
			// Get the x value.
			string searchString = "key=\"d0\">";
			string endString = "</data>";
			
			size_t startPos = m_currentLine.find(searchString) + searchString.length();
			size_t endPos = m_currentLine.find(endString);
			
			string value = m_currentLine.substr(startPos, endPos - startPos);
			m_currentNode.m_x = stof(value);
			
			// Get the y value from the next line.
			getline(graphMlFile, m_currentLine);
			
			searchString = "key=\"d1\">";
			
			startPos = m_currentLine.find(searchString) + searchString.length();
			endPos = m_currentLine.find(endString);
			
			value = m_currentLine.substr(startPos, endPos - startPos);
			m_currentNode.m_y = stof(value);
		}
		// If the line contains "</node>", the node data reading is complete.
		else if (m_currentLine.find("</node>", 0) != string::npos)
		{
			break;
		}
	} while (getline(graphMlFile, m_currentLine));
}

// Read edge data from the file and update connectedNodes of nodes accordingly.
void ReadFile::readEdgeDataFromFile()
{
	// Continue reading until reaching the end of the current edge data.
	do
	{
		// If the line contains "<edge", extract the source and target node IDs.
		if (m_currentLine.find("<edge", 0) != string::npos)
		{
			string searchString = "source=\"";
			string endString = "\" target=\"";
			
			size_t startPos = m_currentLine.find(searchString) + searchString.length();
			size_t endPos = m_currentLine.find(endString);
			string value = m_currentLine.substr(startPos, endPos - startPos);
			
			int src = stoi(value);
			
			searchString = "target=\"";
			endString = "\">";
			
			startPos = m_currentLine.find(searchString) + searchString.length();
			endPos = m_currentLine.find(endString);
			value = m_currentLine.substr(startPos, endPos - startPos);
			
			// Update the connectedNodes list for the source node.
			m_nodes[src].connectedNodes.push_back(stoi(value));
		}
		// If the line contains "</edge>", the edge data reading is complete.
		else if (m_currentLine.find("</edge>", 0) != string::npos)
		{
			break;
		}
	} while (getline(graphMlFile, m_currentLine));
}

// Set a new filename for reading graph data.
void ReadFile::setFilename(string filename)
{
	this->m_filename = filename;
}

// Get the map of nodes read from the file.
unordered_map<uint16_t, NodeData> ReadFile::getNodes()
{
	return this->m_nodes;
}

// Export the node data to an inline file (not implemented).
void ReadFile::exportMapToInl()
{
	// Not implemented yet.
}

\end{lstlisting}


\subsection{unittest\_readFile.cpp}
\begin{lstlisting}
#include <gtest/gtest.h>
#include "../car_readFile/car_readFile.hpp"

class carReadFile_Fixture : public ::testing::Test, ReadFile
{
	protected:
	ReadFile m_ReadFile;
	
	vector<NodeData> m_testNodes;
};

TEST_F(carReadFile_Fixture, readTest1)
{
	// set test Variables
	NodeData testData;
	testData.m_id = 1;
	testData.m_x = stof("2.22");
	testData.m_y = stof("2.8");
	testData.connectedNodes.push_back(3);
	m_testNodes.push_back(testData);
	
	m_ReadFile.setFilename("unittest_UselessTest-Graph.graphml");
	EXPECT_NO_THROW(m_ReadFile.readFile());
	
	unordered_map<uint16_t, NodeData> actualNodes = m_ReadFile.getNodes();
	
	EXPECT_EQ(actualNodes[1U].m_id, testData.m_id);
	EXPECT_EQ(actualNodes[1U].m_x, testData.m_x);
	EXPECT_EQ(actualNodes[1U].m_y, testData.m_y);
	EXPECT_EQ(actualNodes[1U].connectedNodes[0], testData.connectedNodes[0]);
}

TEST_F(carReadFile_Fixture, readTest2)
{
	
	// set test Variables
	unordered_map<uint16_t, NodeData> testNodes;
	
	testNodes[1U] = NodeData(1, stof("2.22"), stof("2.8"), {3, 4});
	testNodes[2U] = NodeData(2, stof("2.22"), stof("3.17"), {3});
	testNodes[3U] = NodeData(3, stof("2.79"), stof("3.74"), {});
	testNodes[4U] = NodeData(4, stof("3.16"), stof("3.74"), {1});
	
	m_ReadFile.setFilename("unittest_Test-Graph.graphml");
	EXPECT_NO_THROW(m_ReadFile.readFile());
	
	unordered_map<uint16_t, NodeData> actualNodes = m_ReadFile.getNodes();
	
	for (uint16_t cnt{1U}; cnt < 4U; ++cnt)
	{
		EXPECT_EQ(actualNodes[cnt].m_id, testNodes[cnt].m_id);
		EXPECT_EQ(actualNodes[cnt].m_x, testNodes[cnt].m_x);
		EXPECT_EQ(actualNodes[cnt].m_y, testNodes[cnt].m_y);
		
		for (uint16_t i{0}; i < actualNodes[cnt].connectedNodes.size(); ++i)
		{
			EXPECT_EQ(actualNodes[cnt].connectedNodes[i], testNodes[cnt].connectedNodes[i]);
		}
	}
}
\end{lstlisting}

\section{Code des implementierten Algorithmus}
Hier finden Sie den vollst�ndigen Quellcode der Implementierung des A*-Algorithmus in der Programmiersprache C++. Der Quellcode enth�lt Kommentare und Erkl�rungen, die den Algorithmus und seine Funktionsweise veranschaulichen. Dies erm�glicht es Ihnen, den Algorithmus zu studieren, zu analysieren und bei Bedarf anzupassen oder zu erweitern.
\subsection{car\_Algorithm.hpp}
\begin{lstlisting}
#ifndef CARALGORITHM_HPP
#define CARALGORITHM_HPP

#include <vector>
#include <queue>
#include <unordered_map>
#include <cmath>

#include "../car_readFile/car_readFile.hpp"

using namespace std;

// Tolerance value for comparing floating-point numbers.
const float toleranz = 0.03f;

// Represents a node used in the A* algorithm.
struct Node
{
	int id;         // The ID of the node in the graph.
	float fScore;   // The f-score value used in the A* algorithm for node comparison.
	float gScore;   // The g-score value used in the A* algorithm for tracking the path cost.
	
	// Custom comparison operator used for priority queue ordering.
	bool operator<(const Node& other) const { return fScore > other.fScore; }
};

// The A* algorithm class that inherits from the ReadFile class.
class AStarAlgorithm : ReadFile
{
	public:
	// Executes the A* algorithm to find the shortest path between two nodes.
	void aStarAlgorithm(int startId, int goalId);
	
	// Shortens straight road segments by removing intermediate nodes with similar positions.
	void shortenStraightRoad();
	
	// Returns the path found by the A* algorithm.
	vector<uint16_t> getPath();
	
	// Returns a reference to the ReadFile object used for reading input data.
	ReadFile& getReadFile();
	
	private:
	ReadFile m_readFile;             // Object to read input data from a file.
	vector<uint16_t> m_path;         // The final path found by the A* algorithm.
	
	// Calculates the heuristic value (estimated cost) between two nodes A and B.
	// This heuristic is used to guide the A* algorithm towards the goal efficiently.
	float heuristic(NodeData& nodeA, NodeData& nodeB);
	
	// Checks if the x-coordinate of two points is within the tolerance range.
	bool tolerancX(float currX, float prevX);
	
	// Checks if the y-coordinate of two points is within the tolerance range.
	bool tolerancY(float currY, float prevY);
};

#endif
\end{lstlisting}

\subsection{car\_Algorithm.cpp}
\begin{lstlisting}
#include "car_Algorithm.hpp"

vector<uint16_t> AStarAlgorithm::getPath()
{
	// Return the final path found by the A* algorithm.
	return this->m_path;
}

ReadFile& AStarAlgorithm::getReadFile()
{
	// Return a reference to the ReadFile object used for reading input data.
	return this->m_readFile;
}

// Calculate the heuristic value (estimated cost) between two nodes A and B.
// This heuristic is used to guide the A* algorithm towards the goal efficiently.
float AStarAlgorithm::heuristic(NodeData& nodeA, NodeData& nodeB)
{
	return sqrt(pow(nodeA.m_x - nodeB.m_x, 2) + pow(nodeA.m_y - nodeB.m_y, 2));
}

// Check if the x-coordinate of two points is within the tolerance range.
bool AStarAlgorithm::tolerancX(float currX, float prevX)
{
	return (((currX - toleranz) <= prevX) && (prevX <= (currX + toleranz)));
}

// Check if the y-coordinate of two points is within the tolerance range.
bool AStarAlgorithm::tolerancY(float currY, float prevY)
{
	return (((currY - toleranz) <= prevY) && (prevY <= (currY + toleranz)));
}

void AStarAlgorithm::aStarAlgorithm(int startId, int goalId)
{
	// Create two hash maps to store the gScore and fScore for each node.
	unordered_map<uint16_t, float> gScores;
	unordered_map<uint16_t, uint16_t> cameFrom;
	priority_queue<Node> openSet;
	
	gScores[startId] = 0.0f;
	openSet.push({ startId, heuristic(m_readFile.getNodes()[startId], m_readFile.getNodes()[goalId]), 0.0f });
	
	while (!openSet.empty())
	{
		Node current = openSet.top();
		if (current.id == goalId)
		{
			// Reconstruct the path by backtracking from the goal node to the start node.
			int node = current.id;
			m_path.push_back(node);
			while (cameFrom.find(node) != cameFrom.end())
			{
				node = cameFrom[node];
				m_path.push_back(node);
			}
			// Reverse the path to get the correct order.
			reverse(m_path.begin(), m_path.end());
			// Path found, exit the loop.
			break;
		}
		
		openSet.pop();
		vector<int> currentIdConnectedNodes = m_readFile.getNodes()[current.id].connectedNodes;
		for (const auto& neighborId : currentIdConnectedNodes)
		{
			float tentativeGScore = gScores[current.id] + heuristic(m_readFile.getNodes()[current.id], m_readFile.getNodes()[neighborId]);
			if (gScores.find(neighborId) == gScores.end() || tentativeGScore < gScores[neighborId])
			{
				gScores[neighborId] = tentativeGScore;
				float fScore = tentativeGScore + heuristic(m_readFile.getNodes()[neighborId], m_readFile.getNodes()[goalId]);
				openSet.push({ neighborId, fScore, tentativeGScore });
				cameFrom[neighborId] = current.id;
			}
		}
	}
}

void AStarAlgorithm::shortenStraightRoad()
{
	// Remove intermediate nodes in the path that are within the tolerance of straight road segments.
	for(uint16_t i{1U}; i < (m_path.size() - 1); ++i)
	{
		if(tolerancX(m_readFile.getNodes()[m_path.at(i)].m_x, m_readFile.getNodes()[m_path.at(i + 1)].m_x)
		|| tolerancY(m_readFile.getNodes()[m_path.at(i)].m_y, m_readFile.getNodes()[m_path.at(i + 1)].m_y))
		{
			m_path.erase(m_path.begin() + i);
		}
	}
}
\end{lstlisting}

\subsection{unittest\_Algorithm.cpp}
\begin{lstlisting}
#include <gtest/gtest.h>
#include "../car_algorithm/car_Algorithm.hpp"
	
class carAlgorithm_Fixture : public ::testing::Test
{
	protected:
	AStarAlgorithm m_AStarAlgorithm;
};
	
TEST_F(carAlgorithm_Fixture, simpleAStarTest)
{
	vector<uint16_t> expectedPath = {115, 116, 117, 118};
	
	m_AStarAlgorithm.getReadFile().setFilename("Test_track.graphml");
	m_AStarAlgorithm.getReadFile().readFile();
		
	m_AStarAlgorithm.aStarAlgorithm(115, 118);
		
	vector<uint16_t> actualPath = m_AStarAlgorithm.getPath();
		
	if (actualPath.empty())
	{
		ASSERT_FALSE(true) << "No Path was found";
	}
	for (uint16_t i{0}; i < actualPath.size(); ++i)
	{
		EXPECT_EQ(expectedPath.at(i), actualPath.at(i));
	}
}
	
TEST_F(carAlgorithm_Fixture, simpleAStarTest2)
{
	vector<uint16_t> expectedPath = {57, 58, 59, 2, 9, 7, 74, 75, 16, 20, 17};
	
	m_AStarAlgorithm.getReadFile().setFilename("Test_track.graphml");
	m_AStarAlgorithm.getReadFile().readFile();
	
	m_AStarAlgorithm.aStarAlgorithm(57, 17);
	
	vector<uint16_t> actualPath = m_AStarAlgorithm.getPath();
		
	if (actualPath.empty())
	{
		ASSERT_FALSE(true) << "No Path was found";
	}
	for (uint16_t i{0}; i < actualPath.size(); ++i)
	{
		EXPECT_EQ(expectedPath.at(i), actualPath.at(i));
	}
}
	
TEST_F(carAlgorithm_Fixture, crossParking)
{
	vector<uint16_t> expectedPath = {86, 77, 82, 80, 92, 93, 94, 68, 73, 71, 124, 125, 126, 127, 128, 59, 64, 62, 148, 149, 150, 151, 152, 153, 154, 155, 156, 157, 158, 159, 160, 161, 162, 163};
		
	m_AStarAlgorithm.getReadFile().setFilename("Competition_track.graphml");
	m_AStarAlgorithm.getReadFile().readFile();
		
	m_AStarAlgorithm.aStarAlgorithm(86, 163);
		
	vector<uint16_t> actualPath = m_AStarAlgorithm.getPath();
		
	if (actualPath.empty())
	{
		ASSERT_FALSE(true) << "No Path was found";
	}
	for (uint16_t i{0}; i < actualPath.size(); ++i)
	{
		EXPECT_EQ(expectedPath.at(i), actualPath.at(i));
	}
}
	
\end{lstlisting}