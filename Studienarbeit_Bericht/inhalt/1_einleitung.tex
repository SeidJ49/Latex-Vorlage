\chapter{Einleitung}

In diesem Kapitel wird die Motivation f�r die Wahl dieses Themas erl�utert und die Zielsetzung der Studienarbeit festgelegt. Zudem wird ein �berblick �ber den Aufbau und den Inhalt der Arbeit gegeben.
\section{Motivation}
Die vorliegende Studienarbeit befasst sich mit der Implementierung und Evaluierung des A*-Routenplanungsalgorithmus, einem leistungsf�higen und vielseitig einsetzbaren Routenplanungsalgorithmus. \\
F�r die Wahl dieses Themas gibt es mehrere Gr�nde, die im Folgenden erl�utert werden.

Effiziente Routenplanung spielt in vielen Anwendungsbereichen eine entscheidende Rolle. Von Navigationssystemen und Logistik bis hin zu Robotik und Videospielen sind viele Systeme auf eine schnelle und zuverl�ssige Wegfindung angewiesen. \\
Der A*-Algorithmus hat sich dabei als eines der besten Verfahren erwiesen und wird daher in der Praxis h�ufig eingesetzt.

Die Motivation hinter der Implementierung des A*-Algorithmus ist es, ein grundlegendes Verst�ndnis f�r diesen Algorithmus zu entwickeln und die zugrundeliegenden Konzepte zu erforschen. Die Arbeit soll es erm�glichen, die Funktionsweise des A*-Algorithmus praktisch in C++ umzusetzen und dadruch einen tieferen Einblick in die Arbeitsweise von Routing-Algorithmen zu gewinnen. 
\newpage
Ein weiterer Aspekt dieser Arbeit liegt in der M�glichkeit, die Leistungsf�higkeit des A*-Algorithmus unter verschiedenen Bedingungen zu evaluieren. Dabei sollen sowohl einfache auch komplexe Graphen analysiert werden, um den Einfluss von Graphengr��e und -struktur auf die Laufzeit des Algorithmus zu untersuchen. Der Vergleich mit anderen Routingalgorithmen erlaubt es au�erdem, die Vor- und Nachteile des A*-Algorithmus gegen�ber alternativen Ans�tzen zu diskutieren. 

\section{Zielsetzung der Studienarbeit}
Das Hauptziel dieser Arbeit ist es, den A*-Algorithmus in C++ zu implementieren und seine Funktionsweise zu verstehen. Dabei werden die verschiedenen Komponenten des Algorithmus untersucht, wie z.B. die Heuristik, sowie die Datenstrukturen.

Ein weiterer Aspekt ist die Bewertung der Leistungsf�higkeit des A*-Algorithmus. \\
Es werden verschiedene Tests durchgef�hrt, um die Laufzeit des Algorithmus auf verschiedenen Graphen zu messen und zu vergleichen.

Die Erkenntnisse aus dieser Studienarbeit sollen dazu dienen, die Funktionsweise des A*-Algorithmus besser zu verstehen und seine Leistungsf�higkeit in verschiedenen Anwendungsf�llen zu bewerten. Die Ergebnisse k�nnen wertvolle Informationen liefern, um den A*-Algorithmus in verschiedenen Bereichen effektiv einzusetzen und m�gliche Optimierungsm�glichkeiten aufzuzeigen.

\section{Aufbau der Studienarbeit}
Die vorliegende Studienarbeit befasst sich mit der Implementierung eines A*-Routenplan Algorithmus und untersucht dessen Leistungsf�higkeit in verschiedenen Szenarien. 

In der Einleitung wird die Motivation f�r die Themenwahl erl�utert und die Zielsetzung der Studienarbeit definiert. Hier wird ein �berblick �ber die Problemstellung gegeben und die Vorgehensweise in der Arbeit beschrieben. 

\newpage
Im Abschnitt \glqq Grundlagen \grqq{} werden die grundlegenden Konzepte und Begriffe erl�utert, die f�r das Verst�ndnis des A*-Algorithmus und seiner Implementierung relevant sind. 

Der Abschnitt \glqq Datenstrukturen und Implementierungsdetails\grqq{} befasst sich mit den technischen Aspekten der Implementierung. Hier wird erl�utert, wie eine GraphML-Datei eingelesen wird und wie der A*-Algorithmus konkret implementiert wurde.\\ Es werden relevanten Datenstrukturen und Funktionen erl�utert, die f�r die Implementierung des Algorithmus ben�tigt werden. 

Die \glqq Evaluierung und Performance-Analyse\grqq{} bilden den vierten Abschnitt der Arbeit. Hier werden verschiedene Testf�lle vorgestellt, die das Einlesen der GraphML-Datei und die Implementierung des Algorithmus �berpr�fen. Au�erdem wird ein Vergleich der Laufzeiten des A*-Algorithmus auf verschiedenen Graphen durchgef�hrt, um die Performance in verschiedenen Szenarien zu bewerten. 

Im abschlie�enden Abschnitt werden die Ergebnisse und Erkenntnisse der Studienarbeit zusammengefasst. 

Der Anhang der Studienarbeit enth�lt die Quellcodes der implementierten Funktionen sowie die Unit-Tests zur �berpr�fung der Implementierung. 